\chapter{类型和变量}

\textbf{类型}是数据的一种属性,它定义了数据的含义以及可以对数据执行的操作。类型可以分为内建类型和用户自定义的类型。内建类型是指Berry语言内置的一些基本类型,其中不基于类定义的类型被称为\textbf{简单类型}。基于类定义的类型称为\textbf{类类型},内建类型有一些属于类类型,用户自定义的类型也是类类型。

\section{内建类型}

\subsection{简单类型}

\subsubsection{Nil}

Nil类型即空类型,它表示对象具有一个无效值,也可以说对象没有有意义的值。这是一个十分特殊的类型,尽管我们可能会说一个变量为 \texttt{nil},但实际上nil类型没有值,因此这里说的其实是该变量的类型为nil(而不是值)。

一个变量在赋值之前的默认值就是 \texttt{nil}。该类型可以用于逻辑运算,在逻辑运算中,\texttt{nil} 等价于 \texttt{false}。

\subsubsection{整数类型}

整数类型(integer)表示有符号的整数,简称整数。该类型所表示的整数的比特位数取决于具体实现,在32位平台中通常由32位的有符号整数。整数是一种算术类型并支持所有的算术运算。在使用该类型时要注意整数的取值范围,典型的32位有符号整数取值范围是$-2147483648$到$2147483647$之间。

\subsubsection{实数类型}

实数类型(real),准确地说是浮点类型。实数类型通常实现为单精度浮点数或者双精度浮点数。实数类型也是一种算术类型。相比于整数类型,实数类型有更高的精度和更大的取值范围,因此该类型更适用于数学计算。需要注意的是,实数类型实际上是浮点数,因此还是存在精度问题,例如将两个 \texttt{real} 类型的数值进行相等比较是不推荐的。

整数和实数同时参与运算时通常会将整数转换为实数。

\subsubsection{布尔类型} \label{section::type_bool}

布尔类型(boolean)用于逻辑运算,其具有两个值 \texttt{true} 和 \texttt{false},表示逻辑和布尔代数中的两个真值(真和假)。布尔类型主要用于条件判断。逻辑表达式和关系表达式的操作数以及返回值都是布尔类型,而且 \texttt{if}、\texttt{while} 等语句都使用布尔类型作为条件检测。

很多时候非布尔值也可以当作布尔类型去使用,这是因为解释器会对参数进行隐式类型转换。这也是诸如 \texttt{if} 语句的条件检测表达式可以使用任何类型参数的原因。各种类型到布尔类型转换的规则为:
\begin{itemize}
    \item Nil:转换为 \texttt{false}。
    \item 整数:值为 \texttt{0} 时转换为 \texttt{false},否则转换为 \texttt{true}。
    \item 实数:值为 \texttt{0.0} 时转换为 \texttt{false},否则转换为 \texttt{true}。
    \item 实例:存在方法 \texttt{tobool()} 时将使用该方法的返回值,否则转换为 \texttt{true}。
    \item 其他:转换为 \texttt{true}。
\end{itemize}

\subsubsection{字符串}

一个字符串(string)是由字符构成的序列。从存储上来讲,Berry把字符串分为长字符串和短字符串两种。相同的短字符串在内存中只有一个实例,并且所有的短字符串链接在一个哈希表中,这种设计有利于提高字符串相等比较的性能,并且可以减少内存使用。由于长字符串的使用频率较低,而哈希运算的开销却不少,因此没有将它们链接到哈希表中,,故内存中可能有多个相同的实例。字符串在创建之后是只读的,因此,对字符串进行“修改”将会产生新的字符串,而原来的字符串不会被修改。

Berry并不关心字符的格式或者编码,例如字符串 \texttt{'abc'} 实际上就是字符 \texttt{'a'},\texttt{'b'} 和 \texttt{'c'} 的ASCII码排列而成。因此,若字符串中存在宽字符(字符长度大于1字节)则无法直接统计字符串中字符的个数,使用 \texttt{length()} 函数实际上只能得到字符串的字节数。另外,为了方便和C语言交互,Berry的字符串总是以 \texttt{'\textbackslash 0'} 字符结束,这个特性对于Berry程序来说是透明的。

字符串类型可以比较大小,因此可以用于关系运算。

\subsubsection{函数}

函数(function)是一段被封装且可供调用的代码,一般用于实现特定的功能。函数实际上是一个大类,其中包括了闭包、原生函数、原生闭包等几个子类型。对于Berry代码来说,所有的函数子类型都具有相同的行为。函数在Berry中属于第一类值,因此它们可以作为值进行传递。此外可以通过“匿名函数”这种“字面量”的形式直接在表达式中使用。

函数是一种只读的对象,一旦定义就不能修改。可以比较两个函数是否相等(是否是同一个函数),但函数类型不能不能比较大小。

\textbf{原生函数}(native function)和\textbf{原生闭包}(native closure)是指使用C语言实现的函数和闭包。原生函数和原生闭包的一个主要用途是提供Berry语言本身不提供的功能,例如IO操作和底层操作等。如果一段代码的使用频率很高并且对性能有要求,也建议将其改写为实现为原生函数或原生闭包。

\subsubsection{类}

在面向对象的编程中,类(class)是一个可扩展的程序代码模板。类用于创建实例对象,因此类可以说是实例的“类型”。所有的实例对象其类型都是 \texttt{instance} ,同时它们都有一个对应的类,这个类叫做实例的\textbf{类类型}。简单点说,一个类是表示实例对象类型的值,类是对实例特征的抽象。类也是一种只读对象,一旦定义就不能再被修改。

类只能比较等于和不等,而不能比较大小。

\subsubsection{实例}

实例(instance)是通过类生成的具体化对象,由类生成实例的过程称为 \texttt{实例化}。在面向对象编程中,“实例”通常和“对象”同义。不过为了和非实例对象区分,我们不单独使用“对象”一词,而是使用“实例”或者“实例对象”。Berry的实例总是动态分配的,需要搭配垃圾回收器来使用。除了内存分配,实例化的的过程还需要对实例进行初始化,这个过程由 \texttt{构造函数} 来完成。此外还可以在回收对象的内存之前通过 \texttt{析构函数} 完成对象的销毁。

在内部实现中,实例会包含对类的引用,实例本身只存储成员变量而不存储方法。

\subsection{类类型}

内建类型中有一些属于类类型,它们是 \texttt{list}、\texttt{map} 和 \texttt{range}。与自定义类型不同,内建的类类型可以使用字面量来构造,例如 \texttt{[1, 2, 3]} 是 \texttt{list} 类型的字面量。

\subsubsection{List}

List类是一种容器,该类型提供对列表数据类型的支持。Berry的list是元素的有序集合,且list中每个元素都有唯一的整数索引,可以根据索引直接存取每个元素。List支持在任意位置插入或删除元素,该元素可以是任意类型。除了使用索引,还可以使用迭代器来访问list中的元素。

List的实现方式是动态数组,该数据结构具有很好的随机访问性能。在list尾部增删元素的效率很高,但是在list中间增删元素的效率较低。

List容器的字面值初始化方法是使用方括号包围的对象列表,多个对象之间使用逗号隔开,例如:
\begin{lstlisting}[language=berry, numbers=none]
[]    ['string']    [0, 1, 2, 'list']
\end{lstlisting}

操作:\texttt{insert}、\texttt{push}、\texttt{item}、\texttt{setitem}。

\subsubsection{Map}

Map也是一种容器,map是键—值对构成的集合,每个可能的键最多在集合中出现一次。Map容器提供以下基本操作:
\begin{itemize}
    \item 添加键值对到集合
    \item 从集合中删除键值对
    \item 修改现有键对应的值
    \item 通过键查找对应的值
\end{itemize}

Map使用hash表实现,具有很高的查找效率。增删键值对的操作中如果发生了“重哈希”则会消耗比较多的时间。

Map容器也能使用字面值方式初始化,写法是使用花括号包围键值对列表,键和值之间使用冒号分隔,键值对之间使用逗号分隔。例如:
\begin{lstlisting}[language=berry, numbers=none]
{}    {'str': 'hello'}    {'str': 'hello', 'int': 45, 78: nil}
\end{lstlisting}

\subsubsection{Range}

Range容器表示一个整数区间,它通常用于在一个整数范围中进行迭代。该类型有一个 \texttt{\_\_lower\_\_} 成员和 \texttt{\_\_upper\_\_} 成员,分别表示范围的下界和上界。Range的字面值写法是一对使用..运算符连接的整数:
\begin{lstlisting}[language=berry, numbers=none]
0 .. 10    -5 .. 5
\end{lstlisting}

当Range类用于迭代时,迭代的元素为从下界到上界之间的所有整数值,也包括边界值,例如 \texttt{0..5} 的迭代结果为:
\begin{lstlisting}[language=berry, numbers=none]
0 1 2 3 4 5
\end{lstlisting}
因此要注意,对于范围为$x\ ..\ (x+n)$的range,其迭代次数为$n+1$次。

\section{变量}

变量(variable)是具有名字的存储空间,该存储空间存储的数据或信息称为变量的值。变量名用于在源代码中引用变量。在不同的作用域中,一个变量名可以绑定多个独立变量,但是变量则没有别名。变量的值可以在程序运行过程中随时访问或更改。Berry是一种动态类型语言,因此变量值的类型是在运行时确定的,且变量可以存储任意类型的值。

\subsection{定义变量}

第一种定义变量的方法是使用赋值语句将一个值赋给新的变量名:
\begin{algorithm}
$\bm{variable}$\texttt{ = }$\bm{expression}$
\end{algorithm}\vspace{-0.6em}\\
$\bm{variable}$为变量的名字,变量名是一个标识符(见\ref{section::identifier}节)。$\bm{expression}$是初始化该变量的表达式。
\begin{lstlisting}[language=berry, numbers=none]
a = 1    b = 'str'
\end{lstlisting}
然而这种定义变量的方法存在一些局限,以下面的代码为例:
\begin{lstlisting}[language=berry]
i = 0
do
    i = 1
    print(i)  # 1
end
print(i)      # 1
\end{lstlisting}
例程中的 \texttt{do} 语句构成了内层作用域,我们在第3行的位置修改了变量 \texttt{i} 的值,在第6行离开内层作用域之后 \texttt{i} 的值还是 \texttt{1}。如果我们希望内层作用域的变量 \texttt{i} 是一个独立的变量,使用直接赋值给新变量名来定义变量的方法无法满足要求,因为标识符 \texttt{i} 已经存在于外层作用域。这种情况可以通过 \texttt{var} 关键字来定义变量:
\begin{algorithm}
\texttt{var }$\bm{variable}$ \\
\texttt{var }$\bm{variable}$\texttt{ = }$\bm{expression}$
\end{algorithm}\vspace{-0.6em}\\
使用 \texttt{var} 定义变量存在两种写法:第一种是在 \texttt{var} 关键字后面跟随变量的名字$\bm{variable}$,这种情况下该变量会被初始化为 \texttt{nil} ,另一种写法是在定义变量的同时初始化变量,此时需要提供初始值表达式$\bm{expression}$。使用 \texttt{var} 定义变量有两种可能的结果:如果当前的作用域没有定义$\bm{variable}$的变量则定义并初始化该变量,否则相当于重新初始化该变量。因此,使用 \texttt{var} 定义的变量会屏蔽外层作用域中的同名变量。

现在我们将前面的例子改用 \texttt{var} 关键字来定义变量:
\begin{lstlisting}[language=berry]
i = 0
do
    var i = 1
    print(i)  # 1
end
print(i)      # 0
\end{lstlisting}
从修改后的例程可以发现,变量 \texttt{i} 在内层作用域中的值是 \texttt{1},而它在外层作用域中的值为 \texttt{0}。这证明使用 \texttt{var} 关键字之后在内层作用域中定义了新的变量 \texttt{i},并且屏蔽了外层作用域中的同名变量。内层作用域结束后,标识符 \texttt{i} 再一次和外层作用域中的变量 \texttt{i} 绑定。

使用 \texttt{var} 关键字定义变量时还可以使用多个变量名构成的列表,变量名之间使用逗号分隔。定义变量时还可以为一个或多个变量进行初始化:
\begin{lstlisting}[language=berry, numbers=none]
var a = 0, b, c = 'test'
\end{lstlisting}

\subsection{作用域和生命周期} \label{section::scope_life}

前面已经提到,变量名可以和多个变量实体(存储空间)绑定,而变量名在每个位置都只和一个实体绑定。需要根据变量名出现的位置来确定变量名所绑定的实体。

\textbf{作用域}(scope)是指名称和实体唯一绑定的代码区域。在作用域之外,名称可能绑定其他的实体,或者不绑定任何实体。实体只在与名称绑定的范围内可见,即变量仅在其作用域内有效。

一个代码块(见\ref{section::block})就是一个作用域。一个变量只在块内部可用,而名字在不同的块中则可能绑定不同的变量实体。下面的例子演示了变量的作用域:
\begin{lstlisting}[language=berry]
var i = 0
do
    var j = 'str'
    print(i, j)  # 0 str
end
# The variable j is not available here
print(i)         # 0
\end{lstlisting}
在这个例程中定义了名称 \texttt{i} 和 \texttt{j}。其中名称 \texttt{i} 定义于 \texttt{do} 语句外,这种定义于最外层块的名称具有\textbf{全局作用域}(global scope)。具有全局作用域作用域的名称自定义之后在整个程序中都可用。名称 \texttt{j} 定义在 \texttt{do} 语句内的块中,这类定义在非最外层块中的名称其具有\textbf{局部作用域}(local scope)。具有局部作用域的名字在作用域以外不能访问。

Berry 有一些内建(built-in)的对象,这些对象都处于全局作用域中。不过内建对象和脚本中定义的全局变量并不处于同一种全局作用域中,内建对象实际上属于\textbf{内建作用域}(built-in scope)。该作用域和普通的全局作用域一样全局可见,但是却能够被普通的全局作用域覆盖。内建对象包括标准库中的函数和类,这些对象包括 \texttt{print}函数、\texttt{type} 函数 和 \texttt{map} 类等。与其他的作用域不同,内建作用域中的变量都是只读的,因此对内建作用域中的变量进行“赋值”实际上是在全局作用域中定义了同名的变量,该变量覆盖了内建作用域中的符号。

\subsubsection{嵌套作用域}

嵌套作用域是指作用域中包含另一个作用域。我们把被包含的作用域称为\textbf{内层作用域},而包含内层作用域的作用域称为\textbf{外层作用域}。外层作用域中定义的名字可以在所有内层作用域中访问。内层作用域还可以重新绑定在外层作用域中已经定义的名字。前面使用 \texttt{var} 定义变量的例程就描述了这一情景。

\subsubsection{变量的生命周期}

程序运行时没有变量名的概念,此时变量以实体的形式存在。变量在程序执行期间的“有效期”就是变量的\textbf{生命周期}。运行时的变量只有在作用域内才是有效的,离开作用域以后变量会被销毁以回收资源。

全局作用域中定义的变量称为\textbf{全局变量},具有\textbf{静态生命周期},这类变量在整个程序运行过程中都可访问且不会被销毁。在局部作用域中定义的变量称为\textbf{局部变量},具有\textbf{动态生命周期},这类变量在离开定义域以后就无法访问并且会被销毁。

由于生命周期不同,局部变量和全局变量使用不同的方式来分配存储空间。局部变量在一种称为\textbf{栈}(stack)的结构上分配,基于栈分配的对象可以在作用域结束时快速回收。全局变量在\textbf{全局表}(global table)中分配,全局表中的对象一旦创建就不会回收,且该表在程序的任何位置都可以访问。

\subsection{变量的类型}

Berry 在运行时确定变量的类型,换言之,变量能存储任何类型的值。因此 Berry 是一种\textbf{动态类型}的语言。解释器并不会在编译期推导变量的类型,这可能导致一些错误会在运行时才暴露,例如执行表达式 \texttt{'1' + 1} 产生的错误是运行时错误而不是编译器错误。采用动态类型的好处是可以简化很多设计,同时程序也会比较灵活,更不必去设计复杂的类型推导系统。

由于解释器对类型缺乏检查,用户代码可能需要自行判断值的类型,也可以利用这个特性来实现一些特殊的操作,这种特性也使重载函数没有存在的必要。例如原生函数 \texttt{type} 即可接受任意类型的参数并返回描述参数类型的字符串。
